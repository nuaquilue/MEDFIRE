\PassOptionsToPackage{unicode=true}{hyperref} % options for packages loaded elsewhere
\PassOptionsToPackage{hyphens}{url}
%
\documentclass[
]{article}
\usepackage{lmodern}
\usepackage{amssymb,amsmath}
\usepackage{ifxetex,ifluatex}
\ifnum 0\ifxetex 1\fi\ifluatex 1\fi=0 % if pdftex
  \usepackage[T1]{fontenc}
  \usepackage[utf8]{inputenc}
  \usepackage{textcomp} % provides euro and other symbols
\else % if luatex or xelatex
  \usepackage{unicode-math}
  \defaultfontfeatures{Scale=MatchLowercase}
  \defaultfontfeatures[\rmfamily]{Ligatures=TeX,Scale=1}
\fi
% use upquote if available, for straight quotes in verbatim environments
\IfFileExists{upquote.sty}{\usepackage{upquote}}{}
\IfFileExists{microtype.sty}{% use microtype if available
  \usepackage[]{microtype}
  \UseMicrotypeSet[protrusion]{basicmath} % disable protrusion for tt fonts
}{}
\makeatletter
\@ifundefined{KOMAClassName}{% if non-KOMA class
  \IfFileExists{parskip.sty}{%
    \usepackage{parskip}
  }{% else
    \setlength{\parindent}{0pt}
    \setlength{\parskip}{6pt plus 2pt minus 1pt}}
}{% if KOMA class
  \KOMAoptions{parskip=half}}
\makeatother
\usepackage{xcolor}
\IfFileExists{xurl.sty}{\usepackage{xurl}}{} % add URL line breaks if available
\IfFileExists{bookmark.sty}{\usepackage{bookmark}}{\usepackage{hyperref}}
\hypersetup{
  pdftitle={MEDFIRE},
  pdfauthor={User's guide v.1},
  pdfborder={0 0 0},
  breaklinks=true}
\urlstyle{same}  % don't use monospace font for urls
\usepackage[margin=1in]{geometry}
\usepackage{color}
\usepackage{fancyvrb}
\newcommand{\VerbBar}{|}
\newcommand{\VERB}{\Verb[commandchars=\\\{\}]}
\DefineVerbatimEnvironment{Highlighting}{Verbatim}{commandchars=\\\{\}}
% Add ',fontsize=\small' for more characters per line
\usepackage{framed}
\definecolor{shadecolor}{RGB}{248,248,248}
\newenvironment{Shaded}{\begin{snugshade}}{\end{snugshade}}
\newcommand{\AlertTok}[1]{\textcolor[rgb]{0.94,0.16,0.16}{#1}}
\newcommand{\AnnotationTok}[1]{\textcolor[rgb]{0.56,0.35,0.01}{\textbf{\textit{#1}}}}
\newcommand{\AttributeTok}[1]{\textcolor[rgb]{0.77,0.63,0.00}{#1}}
\newcommand{\BaseNTok}[1]{\textcolor[rgb]{0.00,0.00,0.81}{#1}}
\newcommand{\BuiltInTok}[1]{#1}
\newcommand{\CharTok}[1]{\textcolor[rgb]{0.31,0.60,0.02}{#1}}
\newcommand{\CommentTok}[1]{\textcolor[rgb]{0.56,0.35,0.01}{\textit{#1}}}
\newcommand{\CommentVarTok}[1]{\textcolor[rgb]{0.56,0.35,0.01}{\textbf{\textit{#1}}}}
\newcommand{\ConstantTok}[1]{\textcolor[rgb]{0.00,0.00,0.00}{#1}}
\newcommand{\ControlFlowTok}[1]{\textcolor[rgb]{0.13,0.29,0.53}{\textbf{#1}}}
\newcommand{\DataTypeTok}[1]{\textcolor[rgb]{0.13,0.29,0.53}{#1}}
\newcommand{\DecValTok}[1]{\textcolor[rgb]{0.00,0.00,0.81}{#1}}
\newcommand{\DocumentationTok}[1]{\textcolor[rgb]{0.56,0.35,0.01}{\textbf{\textit{#1}}}}
\newcommand{\ErrorTok}[1]{\textcolor[rgb]{0.64,0.00,0.00}{\textbf{#1}}}
\newcommand{\ExtensionTok}[1]{#1}
\newcommand{\FloatTok}[1]{\textcolor[rgb]{0.00,0.00,0.81}{#1}}
\newcommand{\FunctionTok}[1]{\textcolor[rgb]{0.00,0.00,0.00}{#1}}
\newcommand{\ImportTok}[1]{#1}
\newcommand{\InformationTok}[1]{\textcolor[rgb]{0.56,0.35,0.01}{\textbf{\textit{#1}}}}
\newcommand{\KeywordTok}[1]{\textcolor[rgb]{0.13,0.29,0.53}{\textbf{#1}}}
\newcommand{\NormalTok}[1]{#1}
\newcommand{\OperatorTok}[1]{\textcolor[rgb]{0.81,0.36,0.00}{\textbf{#1}}}
\newcommand{\OtherTok}[1]{\textcolor[rgb]{0.56,0.35,0.01}{#1}}
\newcommand{\PreprocessorTok}[1]{\textcolor[rgb]{0.56,0.35,0.01}{\textit{#1}}}
\newcommand{\RegionMarkerTok}[1]{#1}
\newcommand{\SpecialCharTok}[1]{\textcolor[rgb]{0.00,0.00,0.00}{#1}}
\newcommand{\SpecialStringTok}[1]{\textcolor[rgb]{0.31,0.60,0.02}{#1}}
\newcommand{\StringTok}[1]{\textcolor[rgb]{0.31,0.60,0.02}{#1}}
\newcommand{\VariableTok}[1]{\textcolor[rgb]{0.00,0.00,0.00}{#1}}
\newcommand{\VerbatimStringTok}[1]{\textcolor[rgb]{0.31,0.60,0.02}{#1}}
\newcommand{\WarningTok}[1]{\textcolor[rgb]{0.56,0.35,0.01}{\textbf{\textit{#1}}}}
\usepackage{graphicx,grffile}
\makeatletter
\def\maxwidth{\ifdim\Gin@nat@width>\linewidth\linewidth\else\Gin@nat@width\fi}
\def\maxheight{\ifdim\Gin@nat@height>\textheight\textheight\else\Gin@nat@height\fi}
\makeatother
% Scale images if necessary, so that they will not overflow the page
% margins by default, and it is still possible to overwrite the defaults
% using explicit options in \includegraphics[width, height, ...]{}
\setkeys{Gin}{width=\maxwidth,height=\maxheight,keepaspectratio}
\setlength{\emergencystretch}{3em}  % prevent overfull lines
\providecommand{\tightlist}{%
  \setlength{\itemsep}{0pt}\setlength{\parskip}{0pt}}
\setcounter{secnumdepth}{-2}
% Redefines (sub)paragraphs to behave more like sections
\ifx\paragraph\undefined\else
  \let\oldparagraph\paragraph
  \renewcommand{\paragraph}[1]{\oldparagraph{#1}\mbox{}}
\fi
\ifx\subparagraph\undefined\else
  \let\oldsubparagraph\subparagraph
  \renewcommand{\subparagraph}[1]{\oldsubparagraph{#1}\mbox{}}
\fi

% set default figure placement to htbp
\makeatletter
\def\fps@figure{htbp}
\makeatother


\title{MEDFIRE}
\author{User's guide v.1}
\date{22 March 2020}

\begin{document}
\maketitle

\hypertarget{the-medfire-model}{%
\section{The MEDFIRE model}\label{the-medfire-model}}

\hypertarget{models-aim}{%
\subsection{Model's aim}\label{models-aim}}

MEDFIRE is a landscape dynamic model that integrates the main factors of
change in Mediterranean forest landscapes. It allows exploring the
spatial and temporal interactions between land-cover changes, fire
regime, forest managment, fire management and vegetation dynamics. The
main model's aim is to generate spatially explicit scenarios of
urban-agro-forest landscape change corresponding to pre-designed
scenario storylines.

Each model scenario dictates which ecological and anthropogenic
processes are active. Though, the chronological order of the processes
included in the model is fixed as follows:

\begin{enumerate}
\def\labelenumi{\arabic{enumi}.}
\item
  Land-cover changes
\item
  Forest management
\item
  Fires and fire suppression
\item
  Prescribed burns
\item
  Drought
\item
  Post-fire regeneration
\item
  Cohort establishment after drought-induced mortality
\item
  Afforestation (i.e.~colonization of scrublands by tree species)
\item
  Forest growth and aging
\end{enumerate}

If applies, climatic data is updated at the beginning of the time step,
before any other process happens.

\hypertarget{working-directory-structure}{%
\subsection{Working directory
structure}\label{working-directory-structure}}

MEDFIRE is a spatially explicit model implemented in R and structured in
the following folders:\\
- \textbf{inputfiles} contains the text files with the current version
of model's parameters.\\
- \textbf{inputlyrs} contains the initial state in raster and Rdata
format of the model.\\
- \textbf{mld} contains the R-scripts files of the model, sub-modules
and reporting functions.\\
- \textbf{output} contains one sub-folder for each scenario, and for
each scenario \emph{scn.def.r} and \emph{scn.custom.def.r} describe the
initialization of model's parameters.\\
- \textbf{rscripts} contains auxiliar R-scripts to run the model, build
model's inputs and analyze model's outputs.

\hypertarget{functions-of-the-model}{%
\subsection{Functions of the model}\label{functions-of-the-model}}

The current version of the model is implemented thought the following
functions (in alphabetic order):

\begin{itemize}
\tightlist
\item
  \textbf{afforestation(\ldots{})} simulates the colonization of
  shrublands by tree forest species according to the age of the shurb,
  the slope, the percentage of mature forest in a neihbourhood (only if
  they are within their potential climatic niche).\\
\item
  \textbf{auxiliars(\ldots{})} contains several simple functions used to
  count cells fullfiling a certain criteria. These functions are used by
  main functions of the model.\\
\item
  \textbf{cohort.establish(\ldots{})} simulates forest regeneration
  after drought-induced mortality. It depends on a secondary species
  matrix (accounting for the presence of secondary species as function
  of dominant species in Mediterranean forest ecosystems) and the
  percentage of forest species in a neighborhood.\\
\item
  \textbf{define.scenario(\ldots{})} initializes the scenario parameters
  and the global variables of the model.\\
\item
  \textbf{drought(\ldots{})} simulates drought-induced mortality of
  forest species falling out their potential climatic niche.\\
\item
  \textbf{fire.regime(\ldots{})} simulates fire events under three
  synoptic weather conditions. Fires spread as wind-driven, convective
  or topographic fires.\\
\item
  \textbf{forest.management(\ldots{})} simulates sylvicultural practices
  (e.g.~thining, clear-cutting) and implements a set of rules based on
  the forest species, biomass, site quality index and forest age.\\
\item
  \textbf{growth(\ldots{})} simulates vegetation productivity (increment
  of biomass) and ageing.\\
\item
  \textbf{land.cover.change(\ldots{})} simulates land-cover transitions
  (e.g.~urbanization, rural abandonment, agriculture conversion)
  following a demand allocation approach.\\
\item
  \textbf{land.dyn.mdl(\ldots{})} is the \textbf{MEDFIRE} model. The
  function loads the spatial state variables, the initialization of
  model's parameters, creates the scenario output sub-folder and
  schedules the processes, e.g.~land-cover changes, forest managment,
  wildfires, drought, post-disturbance regeneration and vegetation
  dynamics.\\
\item
  \textbf{post.fire(\ldots{})} simulates forest regeneration after fire
  as function of the secondary species matrix and the percentage of
  forest species in a neighborhood.\\
\item
  \textbf{prob.igni(\ldots{})} calculates the probability of fire
  ignition depending on elevation, slope, precipitation, density of
  roads, and land interfaces.\\
\item
  \textbf{read.climatic.vars(\ldots{})} reads climatic raster files at 1
  km resolution according to a climatic scenario (\texttt{rcp45} or
  \texttt{rcp85}) and a climatic model
  (\texttt{KNMI-RACMO22E\_ICHEC-EC-EARTH},
  \texttt{KNMI-RACMO22E\_MOHC-HadGEM2-ES},
  \texttt{SMHI-RCA4\_CNRM-CERFACS-CNRM-CM5},
  \texttt{SMHI-RCA4\_MPI-M-MPI-ESM-LR}, or
  \texttt{SMHI-RCA4\_MOHC-HadGEM2-ES}) and builda a data frame with
  annual minimum temperature and annual total precipitation at 1 ha for
  each decade, from 2010 to 2100.\\
\item
  \textbf{read.sdm(\ldots{})} reads species distribution models, that is
  potential climatic niche according to a climatic scenario and climatic
  model for each decade, from 2010 to 2100.\\
\item
  \textbf{read.state.vars(\ldots{})} initialize state dynamic variables:
  land-cover forest map, vegetation biomass, forest age and time since
  last disturbance.\\
\item
  \textbf{read.static.vars(\ldots{})} initiazlie spatially explicit
  static model variables: cell coordinates, mask, elevation, aspect,
  slope, density of roads per ha, 1 km UTM grid, mask for
  topographic-driven fires, mask for wind-driven fires, probability of
  North wind, probability of North-west wind, and probability of West
  wind.\\
\item
  \textbf{update.clim(\ldots{})} updates climatic data according to the
  selected climatic scenario.\\
\item
  \textbf{update.interface(\ldots{})} updates land interface layer.
\end{itemize}

\hypertarget{running-medfire}{%
\section{Running MEDFIRE}\label{running-medfire}}

\hypertarget{how-to-run-a-test-scenario}{%
\subsection{How to run a test
scenario}\label{how-to-run-a-test-scenario}}

\begin{enumerate}
\def\labelenumi{\arabic{enumi}.}
\item
  Clone or download the Github repository
  \href{https://github.com/nuaquilue/MEDFIRE}{MEDFIRE} in a local
  folder, e.g.~C:/WORK/MEDFIRE.
\item
  In R, clean the working space.
\end{enumerate}

\begin{Shaded}
\begin{Highlighting}[]
\KeywordTok{rm}\NormalTok{(}\DataTypeTok{list =} \KeywordTok{ls}\NormalTok{())}
\end{Highlighting}
\end{Shaded}

\begin{enumerate}
\def\labelenumi{\arabic{enumi}.}
\setcounter{enumi}{2}
\tightlist
\item
  Set the working directory to the previous local folder.
\end{enumerate}

\begin{Shaded}
\begin{Highlighting}[]
\KeywordTok{setwd}\NormalTok{(}\StringTok{"C:/WORK/MEDFIRE"}\NormalTok{)}
\end{Highlighting}
\end{Shaded}

\begin{enumerate}
\def\labelenumi{\arabic{enumi}.}
\setcounter{enumi}{3}
\tightlist
\item
  Load the model.
\end{enumerate}

\begin{Shaded}
\begin{Highlighting}[]
\KeywordTok{source}\NormalTok{(}\StringTok{"mdl/landscape.dyn5.r"}\NormalTok{)}
\end{Highlighting}
\end{Shaded}

\begin{enumerate}
\def\labelenumi{\arabic{enumi}.}
\setcounter{enumi}{4}
\tightlist
\item
  Run the model.
\end{enumerate}

\begin{Shaded}
\begin{Highlighting}[]
\KeywordTok{land.dyn.mdl}\NormalTok{(scn.name)}
\end{Highlighting}
\end{Shaded}

\hypertarget{how-to-create-and-run-a-new-scenario}{%
\subsection{How to create and run a new
scenario}\label{how-to-create-and-run-a-new-scenario}}

\begin{enumerate}
\def\labelenumi{\arabic{enumi}.}
\tightlist
\item
  Do steps 2. to 4. and load the function to set the default scenario's
  parameters.
\end{enumerate}

\begin{Shaded}
\begin{Highlighting}[]
\KeywordTok{rm}\NormalTok{(}\DataTypeTok{list =} \KeywordTok{ls}\NormalTok{())}
\KeywordTok{setwd}\NormalTok{(}\StringTok{"C:/WORK/MEDFIRE"}\NormalTok{)}
\KeywordTok{source}\NormalTok{(}\StringTok{"mdl/land.dyn.mdl.r"}\NormalTok{)}
\KeywordTok{source}\NormalTok{(}\StringTok{"mdl/define.scenario.r"}\NormalTok{) }
\end{Highlighting}
\end{Shaded}

\begin{enumerate}
\def\labelenumi{\arabic{enumi}.}
\setcounter{enumi}{1}
\tightlist
\item
  Give a name to the new scenario, e.g.~\emph{``MyTest''}, and call the
  \textbf{define.scenario} fucntion to load model's parameters with the
  default initialization.
\end{enumerate}

\begin{Shaded}
\begin{Highlighting}[]
\NormalTok{scn.name <-}\StringTok{ "MyTest"}
\KeywordTok{define.scenario}\NormalTok{(scn.name)}
\end{Highlighting}
\end{Shaded}

\begin{enumerate}
\def\labelenumi{\arabic{enumi}.}
\setcounter{enumi}{2}
\tightlist
\item
  Customize the initialization of some parameters
\end{enumerate}

\begin{Shaded}
\begin{Highlighting}[]
\NormalTok{file.clim.severity <-}\StringTok{ "ClimaticSeverity_test"}
\NormalTok{file.sprd.weight <-}\StringTok{ "SprdRateWeights_C"}
\NormalTok{nrun <-}\StringTok{ }\DecValTok{3}
\end{Highlighting}
\end{Shaded}

\begin{enumerate}
\def\labelenumi{\arabic{enumi}.}
\setcounter{enumi}{3}
\tightlist
\item
  Write the name of all the above updated parameters in the following
  call of the \textbf{dump} function. It copies these R objects into the
  file \textbf{outputs/MyTst/scn.custom.def.r} (do not change the name
  of this file).
\end{enumerate}

\begin{Shaded}
\begin{Highlighting}[]
\KeywordTok{dump}\NormalTok{(}\KeywordTok{c}\NormalTok{(}\StringTok{"file.clim.severity"}\NormalTok{, }\StringTok{"file.sprd.weight"}\NormalTok{, }\StringTok{"nrun"}\NormalTok{), }
     \KeywordTok{paste0}\NormalTok{(}\StringTok{"outputs/"}\NormalTok{, scn.name, }\StringTok{"/scn.custom.def.r"}\NormalTok{))}
\end{Highlighting}
\end{Shaded}

\begin{enumerate}
\def\labelenumi{\arabic{enumi}.}
\setcounter{enumi}{4}
\tightlist
\item
  Run ths scenario calling the \textbf{land.dyn.mdl} function. A
  sub-folder named \emph{``MyTest''} will be created in the folder
  \emph{outputs}. Model's output files, \textbf{scn.def.r} and
  \textbf{scn.custom.def.r} are saved in it.
\end{enumerate}

\begin{Shaded}
\begin{Highlighting}[]
\KeywordTok{land.dyn.mdl}\NormalTok{(scn.name)}
\end{Highlighting}
\end{Shaded}

\hypertarget{how-to-create-and-run-a-series-of-scenarios}{%
\subsection{How to create and run a series of
scenarios}\label{how-to-create-and-run-a-series-of-scenarios}}

\begin{Shaded}
\begin{Highlighting}[]
\CommentTok{# Clean the workspace}
\KeywordTok{rm}\NormalTok{(}\DataTypeTok{list =} \KeywordTok{ls}\NormalTok{())}

\CommentTok{# Change the working directoyr}
\KeywordTok{setwd}\NormalTok{(}\StringTok{"C:/WORK/MEDFIRE"}\NormalTok{)}

\CommentTok{# Load functions}
\KeywordTok{source}\NormalTok{(}\StringTok{"mdl/land.dyn.mdl.r"}\NormalTok{)}
\KeywordTok{source}\NormalTok{(}\StringTok{"mdl/define.scenario.r"}\NormalTok{) }

\CommentTok{# Recursive run MEDFIRE with customized parameters}
\ControlFlowTok{for}\NormalTok{(i }\ControlFlowTok{in}\NormalTok{ LETTERS[}\DecValTok{1}\OperatorTok{:}\DecValTok{5}\NormalTok{])\{}
\NormalTok{  scn.name <-}\StringTok{ }\KeywordTok{paste0}\NormalTok{(}\StringTok{"TestFire"}\NormalTok{, i)}
  \KeywordTok{define.scenario}\NormalTok{(scn.name)}
\NormalTok{  file.clim.severity <-}\StringTok{ "ClimaticSeverity_test"}
\NormalTok{  file.sprd.weight <-}\StringTok{ }\KeywordTok{paste0}\NormalTok{(}\StringTok{"SprdRateWeights_"}\NormalTok{, i)}
\NormalTok{  nrun <-}\StringTok{ }\DecValTok{3}
  \KeywordTok{dump}\NormalTok{(}\KeywordTok{c}\NormalTok{(}\StringTok{"file.clim.severity"}\NormalTok{, }\StringTok{"file.sprd.weight"}\NormalTok{, }\StringTok{"nrun"}\NormalTok{), }
       \KeywordTok{paste0}\NormalTok{(}\StringTok{"outputs/"}\NormalTok{, scn.name, }\StringTok{"/scn.custom.def.r"}\NormalTok{))}
  \KeywordTok{land.dyn.mdl}\NormalTok{(scn.name)}
\NormalTok{\}}

\CommentTok{# Recursive run MEDFIRE with customized parameters}
\NormalTok{list.scn <-}\StringTok{ }\KeywordTok{paste0}\NormalTok{(}\StringTok{"TestFire_"}\NormalTok{, }\KeywordTok{c}\NormalTok{(}\StringTok{"01"}\NormalTok{, }\StringTok{"02"}\NormalTok{, }\StringTok{"03"}\NormalTok{))}
\NormalTok{rpb <-}\StringTok{ }\KeywordTok{c}\NormalTok{(}\FloatTok{0.5}\NormalTok{,}\DecValTok{1}\NormalTok{,}\FloatTok{1.5}\NormalTok{)}
\ControlFlowTok{for}\NormalTok{(j }\ControlFlowTok{in} \DecValTok{1}\OperatorTok{:}\DecValTok{3}\NormalTok{)\{}
\NormalTok{  scn.name <-}\StringTok{ }\NormalTok{list.scn[j]}
  \KeywordTok{define.scenario}\NormalTok{(scn.name)}
\NormalTok{  file.clim.severity <-}\StringTok{ "ClimaticSeverity_test"}
\NormalTok{  rpb.sr <-}\StringTok{ }\NormalTok{rpb[j]}
  \KeywordTok{dump}\NormalTok{(}\KeywordTok{c}\NormalTok{(}\StringTok{"file.clim.severity"}\NormalTok{, }\StringTok{"rpb.sr"}\NormalTok{), }
       \KeywordTok{paste0}\NormalTok{(}\StringTok{"outputs/"}\NormalTok{, scn.name, }\StringTok{"/scn.custom.def.r"}\NormalTok{))}
  \KeywordTok{land.dyn.mdl}\NormalTok{(scn.name)}
\NormalTok{\}}
\end{Highlighting}
\end{Shaded}

\end{document}
